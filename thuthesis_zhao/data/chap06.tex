\chapter{总结与展望}
\label{cha:intro}

\section{总结}

	针对UEFI BIOS的开发特点与敏捷开发的要求,为了提高软件测试的效率、节省测试的成本、保证测试的进度和质量,本文基于持续集成的思想设计和实现了一套UEFI BIOS自动化测试系统。该套系统能够在每晚对白天修改和扩展过的产品源文件进行集成和测试,完成从产品构建、产品部署、产品测试到产品质量分析的集成测试任务,其中的任何一个环节都是在无人值守的情况下自动完成的。
	
	论文的第一章主要介绍了本文的研究背景、现状、发展动态、选题依据与来源。
	
	论文的第二章主要对集成测试与持续集成测试的概念和发展进行了简介,并且对比和分析了现有的持续集成测试工具。
	
	论文的第三章主要介绍了UEFI BIOS测试工具SCT的设计、实现与使用方法。由于现有的大多数测试工具并不适用于OS之下的UEFI BIOS测试,因此我们需要根据UEFI BIOS的需求,自行研发一款专门用于其测试工作的工具。而SCT也正是UEFI BIOS持续集成测试系统中最为重要的组成之一。
	
	论文的第四章和第五章分别介绍了在X-64架构的Denlow平台和IA-32架构的Nt32平台上,对该两种不同架构、不同平台的UEFI BIOS持续集成测试系统的设计、实现与实验结果。 
	
	目前本文基于持续集成设计和实现的UEFI BIOS自动化测试系统已经成功通过验证,其良好的平台可移植性保证了该系统能够被移植和应用到不同平台开发过程中的持续集成测试任务,例如:Denlow-X64 、 Denlow-IA32、Nt32-IA32、Romley-X64。并且将来还可能移植应用到其他平台,例如:TunnelMountain-X64等。此外该系统具有良好的可扩展性,不仅能够执行产品的构建测试与产品的功能测试,并且能够将其他类型测试加入到该系统中,例如Denlow平台下的LUV测试。
	
	综上所述,本文实现的UEFI BIOS持续集成测试系统具有以下一些优点:
	
	\begin{itemize}
		\item 系统中的任一环节都是自动完成的,有利于减少测试任务中的重复过程所耗费的时间和人力成本。
		\item 系统能够尽早、及时的发现软件不能够构建或者存在产品缺陷的问题,使随时(快速、可靠、低风险)发布功能正常的软件成为了可能。
		\item 系统尽早、尽快进行集成,有助于尽早发现产品中存在的缺陷,避免最终产品集成中出现Bug大量涌现的情况,这样容易及时定位和修正Bug,提高产品开发的质量与效率。
		\item 保证了敏捷开发过程中项目的进度和质量。
	\end{itemize}

\section{展望}

	虽然本文中所设计实现的UEFI BIOS持续集成测试系统有上述优点并且已经实际应用到产品项目的开发中,但是必须承认的是其仍有以下不足之处:
	
	\begin{itemize}
		\item UEFI BIOS持续集成测试系统中使用的SCT对于UEFI BIOS的测试,只能测试出产品的缺陷、定位产品的缺陷,却不能对测试结果进行分析,判断产品缺陷出现的原因,仍然需要开发人员通过测试日志对其进行分析和判断。
		\item UEFI BIOS持续集成测试系统中,后续的一些针对产品的性能、兼容性等测试仍然需要QA去手动完成。
		\item UEFI BIOS持续集成测试系统的鲁棒性仍然有待提高。系统在某些外界异常干扰下运行失败之后,并不能在外界恢复正常后正常运行,而是需要测试人员判断原因重新启动。
		\item UEFI BIOS持续集成测试系统的持续频率有待进一步提升。这需要进一步探索合适的集成频率并且对测试工具进行优化,以减少测试工具在产品测试中耗费的时间成本。
	\end{itemize}
	
	综上所述,以上几个不足之处也正是UEFI BIOS持续集成测试系统进一步工作的需求。